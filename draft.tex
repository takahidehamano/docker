%課題研究レジュメテンプレート ver. 1.2

\documentclass[uplatex]{jsarticle}
\usepackage[top=20mm,bottom=20mm,left=20mm,right=20mm]{geometry}
\usepackage[T1]{fontenc}
\usepackage{txfonts}
\usepackage{wrapfig}
\usepackage[expert,deluxe]{otf}
\usepackage[dvipdfmx,hiresbb]{graphicx}
\usepackage[dvipdfmx]{hyperref}
\usepackage{pxjahyper}
\usepackage{secdot}


\makeatletter
  \renewcommand{\section}{%
    \if@slide\clearpage\fi
    \@startsection{section}{1}{\z@}%
    {\Cvs \@plus.5\Cdp \@minus.2\Cdp}% 前アキ
    {.5\Cvs \@plus.3\Cdp}% 後アキ
    %{\normalfont\Large\headfont\raggedright}}
    {\normalfont\raggedright}}

  \renewcommand{\subsection}{\@startsection{subsection}{2}{\z@}%
    {\Cvs \@plus.5\Cdp \@minus.2\Cdp}% 前アキ
    {.5\Cvs \@plus.3\Cdp}% 後アキ
    %{\normalfont\large\headfont}}
    {\normalfont}}

  \renewcommand{\subsubsection}{\@startsection{subsubsection}{3}{\z@}%
    {\Cvs \@plus.5\Cdp \@minus.2\Cdp}%
    {\z@}%
    %{\normalfont\normalsize\headfont}}
    {\normalfont}}
\makeatother
%ここから上を編集する必要はない.





\title{\vspace{-14mm}不完全情報ゲーム人狼のための人工エージェントと実行環境の構築}
\author{PMコース 矢吹研究室 1342097 浜野太豪}
\date{}%日付を入れる必要はない.
\pagestyle{empty}%ページ番号は振らない.
\begin{document}
\maketitle





\section{研究の背景}

オセロやチェス,将棋,囲碁などのゲーム戦略における研究で,人工知能が用いられてきた.それらのゲームは完全情報ゲームという.完全情報ゲームはすべての意思決定点において,これまでにとられた行動や実現した状態に関する情報がすべて公開されている展開型ゲームのことをいう\cite{okumura2013}.

日本では将棋人工知能が電王戦で注目を集めた.2013年に米長邦雄将棋連盟会長とPuellaαという富士通研究所が開発したコンピュータ将棋ソフトウェアが対局し,Puellaαが勝利した.完全情報ゲームの難しさは,終局までの有効な手数に比例する.具体的にはオセロは10の60乗,チェスは10の120乗,将棋は10の220乗,囲碁は10の360乗と囲碁が一番多い\cite{jinro2}.そのため囲碁は計算量が多いため人間に勝ってはいないが,コンピュータの計算力が上がればいずれコンピュータが勝ることが分かっている.

それに対して,プレイヤーが行動するときにそれまでに起こっている情報をすべて把握することができない不完全情報ゲームが存在する\cite{wiki2}.さらにゲームがコミュニケーションによって進行するコミュニケーションゲームがある.

人狼ゲームはコミュニケーションのみにより勝敗が決定する.不完全情報型のゲームである.将棋や囲碁といった完全情報ゲームとは異なり,多くの情報がプレイヤーによって隠蔽される.各プレイヤーは会話と行動から隠蔽された情報を推測しつつ,自らの情報は隠蔽したままチームの勝利に向けて発言,行動していく.人狼ゲームにはプレイヤーが持つ情報の非対称性,信頼を得る説得・協調行動,嘘を見抜く推論など従来の人工知能分野では扱っていなかった多数の解決すべき問題が存在する\cite{jinro1}.そのため新たな研究が必要である.


\section{研究の目的}

人狼ゲームは従来の完全情報ゲーム型の人工知能分野では扱っていなかった不完全情報型のコミュニケーションゲームであるため,多数の解決すべき問題が存在する.それらの問題を解決するべく人狼エージェントのアルゴリズムを解明する.





\section{プロジェクトマネジメントとの関連}

プロジェクトマネージャが人工知能を業務の中で応用・活用していくには,人工知能プログラミングの知識が必要と考えた.そのため不完全情報ゲーム人工知能プログラミングを研究する.






\section{研究の方法}

本研究ではまず人狼知能プロジェクトから提供されている人狼サーバの起動をする.開発言語はJava,開発環境はEclipseを用いた.人狼エージェントのアルゴリズムを理解するためにサンプル人狼エージェントと第一回人狼知能大会の優勝プログラム饂飩エージェントのソースコードリーディングを行う.

\section{研究の結果}
本研究は以下のように進んでいる.
\begin{enumerate}
\item 人狼知能プロジェクトから提供されたサーバの起動を行った.
\item 人狼知能大会優勝プログラムの饂飩ソースコードを理解した.
\end{enumerate}

調査した結果を以下に示す.
\subsection{饂飩プログラムによる人狼の推論方法}
\subsubsection{確定情報の整理}

大会の配役は,人狼3狂人1村人8霊能者1占い師1狩人1の計15人で行う.
15人中人狼3狂人1のパターンは4650通りである.
4650のパターンから存在しないパターンを消去する.
例えば襲撃されたエージェントがいた場合,襲撃されたエージェントは村人に確定し,
人狼というパターンを削除する.

\subsubsection{人狼の推理}
あらかじめ設定した怪しい行動をとったエージェントに妥当度をつけ残ったパターンのエージェントを評価する.

\subsubsection{確定情報と推理を元に行動を選択}
饂飩プログラムは,以下の得られた情報によって行動を選択する.
\begin{enumerate}
\item  推理で求めた,人狼としての妥当度
\item  エージェントがカミングアウトした役職
\item  エージェントが他のエージェントから受けた判定
\item  当日に宣言された投票数
\end{enumerate}




\subsection{パターン化による人狼の矛盾発見}

饂飩プログラムは,人狼の矛盾をいち早く察知することができる.
占い師はプレイヤーが人狼か人間かを占うことができる.
霊能者は死亡したエージェントが人狼か人間か見ることができる.
たとえばAさんとBさんは占い師を名乗っており,EさんとFさんが霊能者を名乗っており,CさんとDさんが投票によって処刑されたとき

\begin{description}
\item[占い師Aは]Cさんは人狼だった Dさんは人間だった
\item[占い師Bは]Cさんは人間だった Dさんは人狼だった
\item[霊能者Eは]Cさんは人狼だった Dさんは人間だった
\item[霊能者Fは]CさんDさん両方人間だったと答えた.この中に最大3人人狼が紛れ込んでいる.
\end{description}

この時,人狼は占い師Bであることを確定することができる.
なぜなら占い師Bが人間側であった場合
占い師A霊能者E霊能者Fが騙っているため人狼の可能性があり,
さらに占い師Bの占い結果のDさんが人狼のため
占い師Bが人間側である場合,人狼が4人となるため占い師Bが人狼である.

















\section{今後の計画}

第2回人狼知能大会から饂飩プログラムを改良したプログラムが溢れることが予想される.
そのプログラムに勝つことができるアルゴリズムを考察する.そして人狼エージェントをプログラミングする.



\bibliographystyle{junsrt}
\bibliography{biblio}%「biblio.bib」というファイルが必要.

\end{document}
